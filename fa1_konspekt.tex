\documentclass{article}[12pt]
\usepackage[T1]{fontenc}
% font
% \usepackage{clara}
\usepackage{fbb}
\usepackage[utf8]{inputenc}
\usepackage[estonian]{babel}
\usepackage{amsmath}
\usepackage{amssymb}
\usepackage{amsthm}
\usepackage{latexsym}
\usepackage{mathtools}
\usepackage{setspace}
\usepackage{graphicx}
\usepackage{enumitem}
\usepackage{import}
\usepackage{pdfpages}
\usepackage{float}
\usepackage[a4paper, lmargin=0.1666\paperwidth, rmargin=0.1666\paperwidth, tmargin=0.1111\paperheight, bmargin=0.1111\paperheight]{geometry} %margins

% formatting section headings
\usepackage{titlesec}

% for pdf hyperlink functionality
% MUST BE LAST IMPORTED PACKAGE
\usepackage{hyperref}
\hypersetup{
  colorlinks=true,
  linkcolor=black,
  filecolor=magenta,      
  urlcolor=cyan,
}

\titleformat{\section}[block]{\Large\bfseries\filcenter}{\newpage\thesection}{1em}{}[\vspace{2ex}]
\titleformat{\subsection}{\large\bfseries\filcenter}{\thesubsection}{1em}{}[\vspace{1ex}]


\linespread{1.3}

% spacing
\newcommand{\h}{\hspace{0.2cm}}
% set symbols
\newcommand{\R}{\mathbb{R}}
\newcommand{\N}{\mathbb{N}}
\newcommand{\Q}{\mathbb{Q}}
\newcommand{\Z}{\mathbb{Z}}
\newcommand{\K}{\mathbb{K}}
\newcommand{\C}{\mathbb{C}}

% norm
\newcommand{\norm}[1]{||#1||}

\newcommand{\toinf}{\rightarrow\infty}
\newcommand{\limtoinf}[2]{\forall \epsilon > 0 \h \exists N \in \mathbb{N}: k \geq N \Rightarrow \ifthenelse{#2=0}{\left\lvert #1\right\rvert}{\left\lvert #1-#2\right\rvert} <\epsilon}

% Ülesanne
\newtheorem{yl}{Ülesanne}[section]

% Definitsioon
\theoremstyle{definition}
\newtheorem{definition}{Definitsioon}[section]

% Teoreem
\theoremstyle{definition}
\newtheorem{theorem}{Teoreem}[section]

% Lause
\theoremstyle{definition}
\newtheorem{lause}{Lause}[section]

% Järeldus
\newtheorem{jareldus}{Järeldus}[section]

\newtheoremstyle{break}
  {\topsep}{\topsep}%
  {\bfseries}{}%
  {\newline}{}%
\theoremstyle{break}
\newtheorem*{toestus}{Tõestus}

% Muuda default listi stiili
\setlist{noitemsep}
\setlist[1]{labelindent=\parindent}
\setlist[enumerate,1]{label = \arabic*$^\circ$, ref = \arabic*}
\setlist[enumerate,2]{label = \emph{\alph*}),ref 	= \theenumi.\emph{\alph*}}

\begin{document}

\title{Funktsionaalanalüüs 1}
\author{René Piik}
\date{2021 kevad}

\maketitle

\section*{Meta}

Õppejõud: Rainis Haller \newline
Hindamine:
\begin{enumerate}
	\item Kontrolltööd $2\times 20p = 40p$;
	\item Eksam $60p$;
	\item Lisaülesanded.
\end{enumerate}
Semestri lõpus saab ühe kontrolltöödest uuesti teha, kirja läheb sel juhul uuesti kirjutatud töö tulemus.

Käesolev konspekt on koostatud Eve Oja ja Peeter Oja 1991. aastal Tartus ilmunud õpiku "Funktsionaalanalüüs" põhjal.

\tableofcontents

\section{Meetriline ruum, normeeritud ruum ja Banachi ruum}

Hulk $X$, mille puhul peame silmas kindlat kujutust $d: X\times X\rightarrow \R$, on meetriline ruum, kui $\forall x,y,z\in X$ korral:
\begin{enumerate}
	\item $d(x,y)=0 \Leftrightarrow x=y$; \hfill (identsus)
	\item $d(x,y) = d(y,x)$; \hfill (sümmeetria)
	\item $d(x,y) + d(y,z) \geq d(x,z)$. \hfill (kolmnurga võrratus)
\end{enumerate}
Kusjuures viimasest saame tuletada tagurpidi kolmnurga võrratuse.
\begin{align*}
	d(x,y) + d(y,z) &\geq d(x,z) \\
	d(x,y) &\geq d(x,z) - d(y,z) \\
	d(x,y) &\geq \big\lvert d(x,z) - d(y,z) \big\rvert \h\h\text{(seda tuleb eraldi näidata)} \\
\end{align*} 
Saame alati mistahes hulga teisendada meetriliseks ruumiks kasutades kaugust
\[
	d(x,y) = \begin{cases}
		0,\h\h x=y, \\
		1,\h\h x\neq y
	\end{cases}
\]
Näited:
\begin{enumerate}
	\item $X=\R,\h d(x,y) = |x-y|$. Naturaalne kaugus;
	\item $X=\R,\h d(x,y) = |e^x-e^y|$;
	\item $X=\R,\h d(x,y) = |\arctan x - \arctan y|$. See kaugus on huvitav, sest maksimaalne kaugus kahe elemendi vahel on $\pi$. Ei esine lõpmatuid kaugusi.
	\item $X=\R,$
	\[
		d(x,y) = \begin{cases}
			|x-y|,\h\h |x-y| < 1, \\
			1,\h\h |x-y| \geq 1, \\
		\end{cases} = \min \{ |x-y|, 1 \};
	\]
	\item $X=\N,\h d(m,n) = \bigg\lvert \frac{1}{m} - \frac{1}{n} \bigg\rvert$. Seda kaugust kasutati Googles veebilehtede tähtsuste järjestamisel.
	\item Hammingu kaugus.
	Fikseerime $n\in \N$. 
	Olgu $X$ kõikide pikkusega $n$ binaarjärjendite hulk. 
	Hammingu kaugus $d(x,y)$ on vastavate erinevate bittide arv.
	\item Olgu $x=\{0,1\}$. Osutub, et kaugus $d$ iga $x,y\in X$ korral, kus
	\[
		d(x,y) = \min \{ k\in \N: x_k \neq y_k \},
	\]
	on kaugus. Samuti
	\[
		d(x,y) = \sum_{k=1}^\infty \frac{ |x_k-y_k| }{2^k}.
	\]
\end{enumerate}

\begin{yl}
	Olgu $X=\R, d(x,y) = |f(x)-f(y)|$. Leida tingimused, mida funktsioon $f$ peab rahuldama, et $d$ oleks kaugus hulgas $X$.	
\end{yl}

\subsection{Koonduvus meetrilises ruumis}

\begin{definition}
	Öeldakse, et jada $x_n\in X$ koondub elemendiks $x\in X$, kui $d(x_n,x)\rightarrow 0$ protsessis $n\rightarrow \infty$.
	Teisisõnu
	\[
		\forall \epsilon > 0 \h\exists N\in \N : n\geq N\Rightarrow d(x_n,x)<\epsilon
	\]	
\end{definition}
\begin{definition}
	Jada $(x_n)$ nimetatakse statsionaarseks, kui leidub $N$ nii, et sellest kohast alates on kõik jada liikmed ($x_N,x_{N+1},\dots$) võrdsed.
\end{definition}
\begin{theorem}
	Statsionaarne jada koondub alati.
\end{theorem}

Järeldused:
\begin{enumerate}
	\item Koonduval jadal leidub täpselt üks piirelement;
	\item Koonduva jada iga osajada koondub samaks piirelemendiks;
	\item (kauguse pidevus) Kui meetrilises ruumis $(X,d)$ $x_n\rightarrow x$ ja $y_n \rightarrow y$, siis $d(x_n,y_n)\rightarrow d(x,y)$.
\end{enumerate}

\begin{yl}
	Tõesta, et diskreetses meetrilises ruumis koonduvad ainult statsionaarsed jadad.
\end{yl}

\subsection{Normeeritud ruum ja Banachi ruum}

\begin{definition}
	Vektorruumi $X$ nimetatakse normeeritud ruumiks, kui igale tema elemendile $x\in X$ on vastavusse seatud kindel reaalarv $||x||$, mida nimetatakse elemendi $x$ normiks, nii, et on täidetud tingimused ($\lambda \in \R$):
	\begin{enumerate}
		\item $||x|| = 0 \Leftrightarrow x=0$;\hfill (samasuse aksioom)
		\item $||\lambda x|| = ||\lambda||\cdot ||x||$;\hfill (homogeensuse aksioom)
		\item $||x+y|| \leq ||x|| + ||y||$.\hfill (kolmnurga võrratus)
	\end{enumerate}	
\end{definition}

\begin{theorem}
	(Põhiteoreem) Iga normeeritud ruum (NR) on meetriline ruum (MR) kaugusega $d(x,y) = ||x-y||$.
\end{theorem}

\begin{jareldus}
	\begin{enumerate}
		\item[]
		\item $\forall x\in X: \norm{x} \geq 0$ ($\norm{x} = \norm{-x}$);
		\item $\bigg\lvert \norm{x} - \norm{y} \bigg\rvert \leq \norm{x+y}$;
		\item (skalaariga korrutamise pidevus) Kui $\alpha_n\rightarrow \alpha$ ($\K$) ja $x_n\rightarrow x$ ($X$), siis $\alpha_n x_n\rightarrow \alpha x$ ($X$);
		\item (liitmise pidevus) Kui $x_n\rightarrow x$ ja $y_n\rightarrow y$, siis $x_n+y_n \rightarrow x+n$.
	\end{enumerate}
\end{jareldus}

Täielikku normeeritud ruumi nimetatase Banachi ruumiks.

\section{Klassikalised lõplikumõõtmelised normeeritud ruumid}

Vaatleme mittetriviaalset VR-i (üle $\K$) $X$. Olgu $d$ kaugus hulgas $X$.
Tarvilik tingimus: iga $M>0$ korral leiduvad punktid $x,y\in X$ nii, et $d(x,y) = M$.

Tarvilik ja piisav tingimus: $d$ on indutseeritud normi poolt parajasti siis, kui kehtivad kaks tingimust:
\begin{enumerate}
	\item $\forall x,y,z\in X :\h d(x,y) = d(x+z,y+z)$;
	\item $\forall \alpha\in \K \h\forall x,y\in X: \h d(x\alpha,y\alpha) = |\alpha|d(x,y)$.
\end{enumerate}
Sel juhul see norm on defineeritud seosega $\norm{x} = d(x,0)$.


Olgu $X=\{(\xi_1,\dots,\xi_n): \xi_k\in\K\}$ kõigi $n$-komponendiliste vektorite hulk.
Tähistame hulga $X$ elemente $x=(\xi_1,\dots,\xi_n),\h y=(\eta_1,\dots,\eta_n),\h z=(\zeta_1,\dots,\zeta_n)$.
Sellises hulgas on mitu võimalust kauguse defineerimiseks, vastavalt erinevad ka ruumide tähised.
\begin{enumerate}
	\item Ruumis $m_n$ defineeritakse kaugus
	\[
		d(x,y) = \max_{1\leq k \leq n} |\xi_k-\eta_k|.
	\]
	Aksioomide kontroll.
	\begin{enumerate}
		\item $d(x,y) = 0 \Leftrightarrow x=y$.
		Tarvilikus on ilmne. 
		Piisavus. 
		Olgu $\max_{1\leq k \leq n} |\xi_k-\eta_k| = 0$. 
		Siis peavad mõlema vektori kõik koordinaadid olema võrdsed, millest järeldub, et ka vektorid ise on võrdsed.
		\item $d(x,y) = d(y,x)$.
		Absoluutväärtuse märgi tõttu pole oluline, kumb punkt võetakse "esimeseks".
		\item $d(x,y) \leq d(x,z) + d(z,y)$.
		Iga indeksi korral
		\begin{align*}
			d(x,y) &= |\xi_k-\eta_k| \\
			&= |\xi_k-\zeta_k+\zeta_k-\eta_k| \\
			&\leq |\xi_k-\zeta_k|+|\zeta_k-\eta_k| \\
			&\leq d(x,z) + d(z,y).
		\end{align*}
		Leides maksimumi $k$ järgi, saamegi kolmnurga võrratuse.
	\end{enumerate}
	\item Olgu $1\leq p \leq \infty$.
	Ruumis $l_p^n$ defineeritakse kaugus
	\[
		d(x,y) = \left( \sum_{k=1}^n |\xi_k - \eta_k|^p \right)^{\frac{1}{p}}.
	\]
	Olulised erijuhud on $l_1^n$, kus $d(x,y) = \sum_{k=1}^n |\xi_k-\eta_k|$ ja $l_2^n$.
	Viimast ruumi tähistatakse ka $\R^n$ või $\C^n$, üldiselt $\K^n$.
	Ruum $\R^n$ on $n$-mõõtmeline eukleidiline ruum.

	Kauguse aksioomide $1^\circ$ ja $2^\circ$ kehtimist pole raske kontrollida.
	Aksioomi $3^\circ$ kehtivus järeldub näiteks ruumis $l_1^n$ absoluutväärtuse või mooduli omadustest, kuid kõrgema astme ruumide korral on see keerulisem.
	Kasutame üldjuhul viimase tingimuse tõestamiseks \textbf{Hölderi võrratust} ja \textbf{Minkowski võrratust}.
\end{enumerate}

\begin{theorem}
	(Hölderi võrratus) Kui $a_k,b_k\in \K,\h k=1,\dots,n$ ja $1<p<\infty$ ning arv $q$ on määratud võrdusega
	\[
		\frac{1}{p} + \frac{1}{q} = 1,
	\]
	siis kehtib võrratus
	\[
		\sum_{k=1}^n |a_k b_k| \leq \left( \sum_{k=1}^n |a_k|^p \right)^{\frac{1}{p}} \cdot \left( \sum_{k=1}^n |b_k|^q \right)^{\frac{1}{q}}.
	\]
	Teisiti saaks kirjutada
	\[
		\norm{(a_1b_1,\dots,a_nb_n)}_1 \leq \norm{(a_1,\dots,a_n)}_p \cdot \norm{(b_1,\dots,b_n)}_q.
	\]
\end{theorem}
\begin{theorem}
	(Minkowski võrratus) Kui $a_k,b_k\in \K,\h k=1,\dots,n$ ja $1<p<\infty$, siis kehtib võrratus
	\[
		\left( \sum_{k=1}^n |a_k + b_k|^p \right)^{\frac{1}{p}} \leq \left( \sum_{k=1}^n |a_k|^p \right)^{\frac{1}{p}} + \left( \sum_{k=1}^n |b_k|^p \right)^{\frac{1}{p}}.
	\]
	\[
		\norm{(a_1,\dots,a_n) + (b_1,\dots,b_n)}_p \leq \norm{(a_1,\dots,a_n)}_p + \norm{(b_1,\dots,b_n)}_p.
	\]
\end{theorem}

\textit{Vaata moodlest lisamaterjalide alt Rainise lemmiktõestust Minkowski võrratusele!}

\begin{yl}
	Näidata, et
	\[
		\lim_{p\rightarrow \infty} \left( \sum_{k=1}^n |\xi_k-\eta_k|^p \right)^{\frac{1}{p}} = \max_{1\leq k\leq n} |\xi_k - \eta_k|.
	\]
	Selle ülesande tulemuse järgi on kaugus ruumis $m_n$ ruumide $l_p^n$ kauguste piirjuht, mistõttu on õigustatud tähistus $m_n = l_\infty^n$.
\end{yl}

\subsection{Jadade koonduvus}

Osutub, et ruumis $(\K_n, \norm{\cdot}_p)$, kus $1\leq p\leq \infty$, $x_n\rightarrow x$\dots

\section{Klassikalised jadaruumid}

Siin vaatleme jadaruume, kus meetrilise ruumi $X$ elementideks on arvjadad $(\xi_k)$, $\xi_k\in\K$.
Tähistame ruumi $X$ elemente $x=(\xi_k),y=(\eta_k),z=(\zeta_k)$.
\begin{enumerate}
	\item Tõkestatud jadade ruum $m$ on hulgana kirjeldatav:
	\[
		m=\{ (\xi_k):\xi_k\in\K,\h\sup_k |\xi_k|<\infty \},
	\]
	s.t. $x=(\xi_k)\in m$ parajasti siis, kui leidub arv $M$ (mis võib sõltuda elemendist $x$) nii, et $|\xi_k|\leq M$.
	Kasutatakse ka tähistust $m=l_\infty$.
	Ruumis $m$ defineeritakse kaugus võrdusega
	\[
		d(x,y) = \sup_k |\xi_k-\eta_k|.
	\]
	Kui $x,y\in m$, siis $\exists M,N\in\R$ nii, et $|\xi_k|\leq M$ ja $|\eta_k|\leq N$ iga $k\in\N$ korral, ning
	\[
		|\xi_k-\eta_k|\leq |\xi_k|+|\eta_k|\leq M+N,
	\]
	seega
	\[
		\sup_k |\xi_k-\eta_k| \leq M+N.
	\]
	\item Koonduvate jadade ruum $c$ on hulk
	\[
		c=\{ (\xi_k):\xi_k\in\K,\h\exists\lim_k \xi_k\in\K \}.
	\]
	Koonduv jada on tõkestatud, s.t. $c\subset m$.
	Kaugus ruumis $c$ defineeritakse nagu ruumis $m$, mistõttu ruum $c$ on ruumi $m$ alamruum.
	\item Nulliks koonduvate jadade ruum $c_0$ on hulk
	\[
		c_0= \{ (\xi_k):\xi_k\in\K,\h\lim_k\xi_k = 0 \}.
	\]
	Kaugus selles ruumis defineeritakse veelkord samamoodi nagu ruumis $m$.
	Seega kehtib $c_0\subset c\subset m$, kusjuures on tegu teineteise alamruumidega.
	\item Olgu $1\leq p < \infty$. 
	Ruum $l_p$ on hulk
	\[
		l_p = \{ (\xi_k):\xi_k\in\K,\h \sum_{k=1}^\infty |\xi_k|^p < \infty \},
	\]
	milles kaugus defineeritakse võrdusega
	\[
		d(x,y) = \left( \sum_{k=1}^\infty |\xi_k-\eta_k|^p \right)^{\frac{1}{p}}.
	\]
	Kauguse aksioomide $1^\circ$ ja $2^\circ$ kehtivus on ilmne.
	Kolmnurga võrratuse põhjendamiseks kasutame kolmnurga võrratust ruumis $l_p^n$, misjärel võtame vasakul piirväärtuse $n\rightarrow \infty$.
	Kehtivad hulgateoreetilised sisalduvused $l_1\subset l_p \subset l_q \subset c_0$, kus $1\leq p \leq q < \infty$, kuid siin ei ole tegemist alamruumidega meetriliste ruumide mõttes.
	Samuti nagu ka $l_p^n$ ja $m_n$ kauguste kohta, saab ka siin tõestada, et
	\[
		\lim_{p\rightarrow \infty} \left( \sum_{k=1}^\infty |\xi_k|^p \right)^{\frac{1}{p}} = \sup_k |\xi_k|,\h (\xi_k)\in \bigcup_{p\geq 1} l_p,
	\]
	mispärast kasutatakse mõnikord tähistust $c_0=l_\infty$.
	\item Kõigi arvjadade ruumis $s = \{ (\xi_k):\xi_k\in \K \}$ antakse kaugus
	\[
		d(x,y) = \sum_{k=1}^\infty \frac{1}{2^k} \frac{ |\xi_k-\eta_k| }{1 + |\xi_k-\eta_k| }.
	\]
	Kuna $\frac{ |\xi_k-\eta_k| }{1 + |\xi_k-\eta_k| } < 1$, siis kauguse definitsioonis esinev rida koondub. Paneme tähele, et iga elemendipaari $x,y$ korral $d(x,y)<1$ ruumis $s$.

	Jällegi on esimese ja teise kauguse aksioomi näitamine lihtne, kolmanda kirjutame välja.
	Olgu funktsioon $\phi(t) = \frac{t}{1+t},\h t\geq 0$. Siis $\phi'(t) = \frac{1}{(1+t)^2}>0$, mistõttu funktsioon $\phi$ on kasvav. 
	Seepärast mistahes $a,b\in\K$ korral
	\begin{align*}
		\frac{ |a+b| }{ 1+|a+b| } &\leq \frac{ |a|+|b| }{ 1+|a|+|b| } \\
		&= \frac{ |a| }{ 1+|a|+|b| } + \frac{ |b| }{ 1+|a|+|b| } \\
		&\leq \frac{ |a| }{ 1+|a| } + \frac{ |b| }{ 1+|b| },
	\end{align*}
	mistõttu
	\begin{align*}
		\frac{ |\xi_k-\eta_k| }{ 1+|\xi_k-\eta_k| } &= \frac{ |\xi_k-\zeta_k+\zeta_k-\eta_k| }{ 1+|\xi_k-\zeta_k+\zeta_k-\eta_k| } \\
		&\leq \frac{ |\xi_k-\zeta_k| }{ 1+|\xi_k-\zeta_k| } + \frac{ |\zeta_k-\eta_k| }{ 1+|\zeta_k-\eta_k| }.
	\end{align*}
	Saadud võrratusi teguriga $\frac{1}{2^k}$ korrutdes ja $k$ järgi summeerides saamegi tulemuseks kolmnurga võrratuse ruumis $s$.
\end{enumerate}

\subsection{Jadade koonduvus}

\begin{enumerate}
	\item (Ruum $m$) Kui $\N\ni x_n=(\xi_k^n)_{k=1}^\infty,\h n=1,2,\dots$ ja $x=(\xi_k)$, siis koondumine $x_n\rightarrow x$ on samaväärne sellega, et $\sup_k |\xi_k^n-\xi_k| \rightarrow_n 0$, mis on omakorda väljendatav järgmiselt:
	\[
		\forall \epsilon > 0 \h \exists N: |\xi_k^n-\xi_k|\leq \epsilon,\h n>N,\h k=1,2,\dots .
	\]
	Kokkuvõttes tähendab koondumine ruumis $m$ koordinaatide ühtlast koondumist.
	\setcounter{enumi}{3}
	\item Kauguse definitsioonis esineva rea koonduvuse põhjendame Minkowski võrratuse abil.
	Kuna
	\begin{align*}
		\left( \sum_{k=1}^\infty |\xi_k-\eta_k|^p \right)^\frac{1}{p} &\leq \left( \sum_{k=1}^\infty |\xi_k|^p \right)^\frac{1}{p} + \left( \sum_{k=1}^\infty |\eta_k|^p \right)^\frac{1}{p} \\
		&\leq \left( \sum_{k=1}^\infty |\xi_k|^p \right)^\frac{1}{p} + \left( \sum_{k=1}^\infty |\eta_k|^p \right)^\frac{1}{p} < \infty, \\
		&\textit{(miks siin sama rida kaks korda on???)}
	\end{align*}
	siis osasummade jada $\sum_{k=1}^n |\xi_k-\eta_k|^p$, $n=1,2,\dots$ on tõkestatud ning rida koondub.
	\begin{lause}
		Kui $x_n=(\xi_k^n)_{k=1}^\infty,\h n=1,2,\dots$ ja $x=(\xi_k)$, siis koondumine $x_n\rightarrow x$ ruumis $l_p$ on samaväärne järgmiste tingimuste samaaegse kehtimisega:
		\begin{enumerate}
			\item $\xi_k^n \rightarrow_n \xi_k,\h k=1,2,\dots$ (koordinaatide koondumine)
			\item $\forall \epsilon > 0\h\exists K: \sum_{k=K+1}^\infty |\xi_k^n|^p<\epsilon,\h n=1,2,\dots$ (read $\sum_{k=1}^\infty |\xi_k^n|^p$ koonduvad ühtlaselt $n$ suhtes).
		\end{enumerate}
	\end{lause}
	\item \begin{lause}
		Koondumine $x_n\rightarrow x$ ruumis $s$ on samaväärne koordinaatide koondumisega.
	\end{lause}
\end{enumerate}

\section{Klassikalised funktsiooniruumid}

\begin{enumerate}
	\item Ruum $M[a,b]$ on kõigi lõigus $[a,b]$ tõkestatud funktsioonide hulk kaugusega
	\[
		d(x,y) = \sup_{a\leq t \leq b} |x(t)-y(t)|
	\]
	Ka siin (nagu ruumi $m$ korral) saab näidata, et sellise definitsiooniga seatakse funktsioonipaarile $x,y$ vastavusse reaalarv.
	\item Ruum $C[a,b]$ on kõigi lõigus $[a,b]$ pidevate funktsioonide hulk, kus kaugus defineeritakse nagu ruumis $M[a,b]$. 
	Selline kauguse definitsioon on korrektne, sest lõigus pidev funktsioon on ka seal tõkestatud.
	Järelikult $C[a,b]\subset M[a,b]$, kusjuures esimene on teise alamruum.
	Kuna lõigus pidev funktsioon saavutab supreemumi, võib kirjutada $x,y\in C[a,b]$ korral
	\[
		d(x,y) = \max_{a\leq t\leq b} |x(t)-y(t)|.
	\]
	\item Ruum $C^n[a,b]$ on kõigi lõigus $n$ korda pidevalt diferentseeruvate funktsioonide hulk, s.t.
	\[
		C^n[a,b] = \{x:x^{(n)}\in C[a,b]\},
	\]
	kaugusega
	\[
		d(x,y) = \sum_{k=0}^n \max_{a\leq t\leq b} |x^{(k)}(t) - y^{(k)}(t)|,
	\]
	kus loeme $x^{(0)}(t) = x(t)$.
	Ruum $C^n[a,b]$ on küll osahulk ruumis $C[a,b]$, aga mitte alamruum, sest tema meetrika erineb ruumi $C[a,b]$ meetrikast.
	\item Olgu $1\leq p < \infty$. Ruum $L_p(a,b)$ on kõigi funktsioonide $x=x(t)$ hulk, mille korral eksisteerib lõplik Lebesgue'i integraal
	\[
		\int_a^b |x(t)|^p dt,
	\]
	kaugusega
	\[
		d(x,y) = \left( \int_a^b |x(t) - y(t)|^p \right)^{\frac{1}{p}}.
	\]
	Võrdsus $x=y$ tähendab ruumis $L_p(a,b)$ seda, et $x(t) = y(t)$ peaaegu kõikjal ruumis $[a,b]$.

	Kauguse esimene aksioom on täidetud, sest samastame needsamad "peaaegu kõikjal ühtivad funktsioonid".
\end{enumerate}

\subsection{Jadade koonduvus}

\begin{enumerate}
	\item Koondumine $x_n\rightarrow x$ ruumis $M[a,b]$ on funktsioonide jada $x_n$ ühtlane koondumine funktsiooniks $x$ lõigul $[a,b]$, s.t.
	\[
		\forall\epsilon>0\h\exists N: n>N\Rightarrow |x_n(t)-x(t)|<\epsilon,\h t\in[a,b].
	\]
\end{enumerate}

\section{Hulgad meetrilistes ruumides}

Olgu $X$ meetriline ruum.

\begin{definition}
	Olgu $a\in X$ ja $r>0$. Hulka
	\[
		B(a,r) = \{ x\in X: d(x,a) < r \}
	\]
	nimetatakse lahtiseks keraks ja hulka
	\[
		\overline{B}(a,r) = \{ x\in X: d(x,a) \leq r \}
	\]
	kinniseks keraks.
	Element $a$ on siinkohal kera keskpunkt ja $r$ selle raadius.
\end{definition}

\begin{lause}
	Kui $r_1\leq r_2$, siis $B(a,r_1)\subset B(a,r_2)$ ja $\overline{B}(a,r_1)\subset \overline{B}(a,r_2)$.
\end{lause}

\begin{definition}
	Hulka $G\subset X$ nimetatakse lahtiseks, kui hulga $G$ igale punktile leidub teda ümbritsev lahtine kera, mis sisaldub hulgas $G$.
\end{definition}

\begin{lause}
	Hulk $G$ on lahtine parajasti siis, kui hulga $G$ igal punktil leidub ümbrus, mis sisaldub hulgas $G$.
\end{lause}

\begin{definition}
	Hulga $A$ elementi $a$ nimetatakse hulga $A$ sisepunktiks, kui punktil $a$ leidub ümbrus, mis sisaldub hulgas $A$.
\end{definition}

\begin{lause}
	Hulk $G$ on lahtine parajasti siis, kui hulga $G$ kõik punktid on tema sisepunktid.
\end{lause}

\begin{lause}
	Lahtine kera on lahtine hulk.
\end{lause}

\begin{lause}
	Mistahes hulga lahtiste hulkade ühend on lahtine.
\end{lause}
\begin{toestus}
	Olgu hulgad $G_\alpha$ lahtised.
	Valime vabalt $x\in \bigcup_\alpha G_\alpha$.
	Siis leidub selline indeks $\alpha_0$, et $x\in G_{\alpha_0}$.
	Hulk $G_{\alpha_0}$ on lahtine, mistõttu leidub $B(x,r)\subset G_{\alpha_0}$.
	Siis aga $B(x,r)\subset \bigcup_\alpha G_{\alpha}$, mis tähendab, et $x$ on hulga $\bigcup_\alpha G_\alpha$ sisepunkt.
	Seega on hulk $\bigcup_\alpha G_\alpha$ lahtine.
\end{toestus}

\begin{lause}
	Lõpliku hulga lahtiste hulkade ühisosa on lahtine.
\end{lause}
Üldiselt ei kehti, et lõpmatu arvu lahtiste hulkade ühend on lahtine.
Näiteks võttes ruumiks $\R$ ja vahemikeks $\bigcap_{n=1}^\infty \left( -\frac{1}{n} ,\frac{1}{n}\right)=\{ 0 \}$.
Hulk $\{ 0 \}$ pole lahtine, sest mistahes raadiuse $r>0$ korral ei sisaldu vastav lahtine kera selles hulgas.

\begin{definition}
	Ruumi punkti nimetatakse hulga $A\subset X$ rajapunktiks, kui tema iga ümbrus sisaldab nii hulga $A$ punkte kui ka tema täiendhulga $X\setminus A$ punkte.
\end{definition}

\begin{definition}
	Hulka nimetatakse kinniseks, kui ta sisaldab kõik oma rajapunktid.
\end{definition}

\begin{theorem}
	Hulk on kinnine parajasti siis, kui tema täiend on lahtine.
\end{theorem}

\begin{lause}
	Mistahes hulga kinniste hulkade ühisosa on kinnine.
\end{lause}

\begin{lause}
	Lõpliku hulga kinniste hulkade ühend on kinnine.
\end{lause}

Leidub ka hulki, mis pole lahtised ega kinnised, näiteks poollõik $(0,1]$ ruumis $\R$.

\begin{theorem}
	Hulk meetrilises ruumis on kinnine parajasti siis, kui iga tema elementidest moodustatud koonduva jada piirelement kuulub sellesse hulka.
\end{theorem}
\begin{toestus}
	Piisavus.
	Olgu hulk $A$ kinnine ja $x_n\rightarrow x,\h x_n\in A$.
	Oletame vastuväiteliselt, et $x\in X\setminus A$, siis hulga $X\setminus A$ lahtisuse tõttu leidub kera $B(x,r)\in X\setminus A$.
	Siis peaks koondumise $d(x_n,x)\rightarrow 0$ tõttu leiduma $N$ nii, et iga $n>N$ korral $x_n\in B(x,r)\subset X\setminus A$.
	See on aga vastuolus sisaldusega $x_n\in A$.
	Seega $x\in A$.

	Tarvilikkus. Olgu $x$ hulga $A$ rajapunkt.
	Siis iga $n\in\N$ korral $B(x,\frac{1}{n})$ sisaldab $A$ punkte. 
	Valime $x_n\in B(x,\frac{1}{n})\cap A$.
	Sellega saame, et $d(x_n,x)<\frac{1}{n}\rightarrow 0$, s.t. $x_n\rightarrow x$.
	Kuna samal ajal $x_n\in A$, siis $x\in A$ ja $A$ on kinnine.
\end{toestus}

\begin{lause}
	Kinnine kera on kinnine hulk.
\end{lause}

\begin{lause}
	Sfäär $S(a,r)=\{ x\in X:d(x,a) = r \}$ on kinnine hulk.
\end{lause}

\begin{theorem}
	Kui $A\subset X$ ja $x\in X$, siis järgmised tingimused on samaväärsed:
	\begin{enumerate}
		\item $x\in \overline{A}$;
		\item $\exists y_n\in A:\h y_n\rightarrow x$;
		\item $\forall \epsilon > 0 \h \exists y\in A:\h d(x,y)<\epsilon$.
	\end{enumerate}
\end{theorem}

\section{Separaablid meetrilised ruumid}

Olgu $X$ meetriline ruum.
\begin{definition}
	Hulk $A\subset X$ on kõikjal tihe (ruumis $X$), kui $\overline{A} = X$.
\end{definition}

\begin{lause}
	Kui $A\subset X$, siis järgmised tingimused on samaväärsed:
	\begin{enumerate}
		\item $A$ on kõikjal tihe;
		\item $\forall x\in X\h\exists y_n\in A:\h y_n\rightarrow x$;
		\item $\forall x\in X\h\forall \epsilon > 0 \h \exists y\in A:\h d(x,y)<\epsilon$.
	\end{enumerate}
\end{lause}

\begin{definition}
	Meetrilist ruumi nimetatakse separaabliks, kui temas leidub kõikjal tihe ülimalt loenduv hulk.
\end{definition}

Ruumid $m_n$ ja $l_p^n$ on separaablid, kõikjal tihedaks leonduvaks hulgaks $A$ võib võtta kõigi ratsionaalsete komponentidega vektorite hulga.
Kui on tegemist reaalsete ruumidega, siis
\[
	A=\{ (\eta_1,\dots, \eta_n):\eta_k\in\Q \},
\]
komplekssel juhul aga
\[
	A=\{ (\eta_1,\dots, \eta_n):\eta_k = \alpha_k + i\beta_k,\h \alpha_k, \beta_k\in\Q \}.
\]
Nende hulkade $A$ kõikjal tihedus ruumides $m_n$ ja $l_p^n$ üldistab võrdust $\overline{\Q} = \R$.

Ainus vaadeldud jadaruumidest, mis pole separaabel, on tõkestatud jadade ruum $m$.

\begin{yl}
	Näitame, et ruum $l_p$ on separaabel.
\end{yl}

Peame selleks näitama, et mistahes ruumi elemendi $x$ ja positiivse $\epsilon$ korral leidub hulga $A$ element, mille kaugus elemendist $x$ on väiksem kui $\epsilon$.
	
Esiteks, hulgad 
\[
	A_n = \{ (\eta_1,\dots, \eta_n,0,\dots):\eta_k\in\Q \}
\]
on loenduvad.
Seega ka hulk $A=\bigcup_{n=1}^\infty A_n$ on loenduv.
Valime vabalt $x=(\xi_k)\in l_p$ ja $\epsilon > 0$.
Olgu arv $n$ selline, et
\[
	\sum_{k=n+1}^\infty | \xi_k|^p < \left(\frac{\epsilon}{2}\right)^p.
\]
Tähistame $x_n = (\xi_1,\dots,\xi_n,0,\dots)$. 
Siis
\[
	d(x,x_n) = \left(\sum_{k=n+1}^\infty | \xi_k|^p\right)^\frac{1}{p} < \frac{\epsilon}{2}.
\]
Seejärel valime
\[
	y = (\eta_1,\dots,\eta_n,0,\dots)\in A_n\subset A
\]
nii, et
\[
	d(x_n,y) = \left(\sum_{k=1}^\infty | \xi_k-\eta_k |^p\right)^\frac{1}{p} < \frac{\epsilon}{2}.
\]
Kuna $d(x,y) \leq d(x,x_n) + d(x_n, y) < \epsilon$, siis $A$ on kõikjal tihe ruumis $l_p$.

Sama mõttekäiguga saab tõestada, et ruumid $c_0, c$ ja $s$ on separaablid.

Funktsionaalruumidest, mida vaatasime, pole separaabel ainult ruum $M[a,b]$.

\begin{yl}
	Näidata, et diskreetne meetriline ruum on separaabel parajasti siis, kui tema elementide hulk on ülimalt loenduv.
\end{yl}

\begin{yl}
	Tõestada, et separaabli ruumi alamruum on separaabel.
\end{yl}

\section{Täielikud meetrilised ruumid ja Banachi ruumid}

\subsection{Cauchy jadad}

\begin{definition}
	Meetrilise ruumi elementide jada $x_n$ nimetatakse Cauchy jadaks ehk fundamentaaljadaks, kui $d(x_n,x_m)\rightarrow 0$ protsessis $n,m\rightarrow 0$.
	\[
		\forall \epsilon > 0\h\exists N\in\N: \h n,m>N\Rightarrow d(x_n,x_m) < \epsilon.
	\]
\end{definition}

\begin{lause}
	Iga koonduv jada on Cauchy jada.
\end{lause}

\subsection{Meetrilised ruumid}

\begin{definition}
	Meetrilist ruumi nimetatakse täielikuks, kui temas iga Cauchy jada koondub.
\end{definition}

Kuna reaalarvude ruumis $\R$ iga Cauchy jada koondub, siis on $\R$ täielik.
Vaatleme ruumi $\Q$ sama kaugusega, mis $\R$-il.
S.t., et $\Q$ on $\R$ alamruum.
Valime $x_n\in \Q$ nii, et $x_n\rightarrow \pi$.
Siis $x_n$ on Cauchy jada, ta koondub ruumis $\R$, aga ei koondu ruumis $\Q$, sest $\pi\notin \Q$. Seega ruum $\Q$ ei ole täielik.

Jada fundamentaalsus on jada sisemine omadus, jada koonduvus on aga seotud tingimata mingi kindla ruumiga.

\begin{yl}
	Näidata, et kui fundamentaaljada $x_n$ mingi osajada koondub elemendiks $x$, siis ka jada $x_n$ koondub samaks elemendiks $x$.
\end{yl}

\begin{theorem}
	Meetrilise ruumi täielik alamruum on kinnine. 
	Täieliku meetrilise ruumi kinnine alamruum on täielik.
\end{theorem}


\begin{toestus}
	\begin{enumerate}
		\item[]
		\item Olgu $A$ meetrilise ruumi täielik alamruum.
		Vaatleme jada $x_n\in A,x_n\rightarrow x$.
		Kuna $(x_n)$ on Cauchy jada ruumis $A$, siis ta koondub ruumis $A$.
		Seega $x\in A$.
		\item Olgu $X$ täielik meetriline ruum ja $A$ tema kinnine alamruum.
		Vaatleme Cauchy jada $(x_n)\in A$.
		Jada $(x_n)$ koondub ruumis $X$.
		Kuna $A$ on kinnine, siis jada $(x_n)$ piirelement kuulub alamruumi $A$.
		Seega koondub $x_n$ ruumis $A$ ehk $A$ on täielik.
	\end{enumerate}
\end{toestus}

Eelnevalt vaadeldud ruumis on kõik täielikud.
Vaatleme näiteks ruume $m_n$ ja $l_p^n$.

Olgu $x_m=(\xi_1^m,\dots,\xi_n^m),\h m\in\N$ Cauchy jada.
Nendes ruumides
\[
	|\xi_1^m-\xi_n^m|\leq d(x_m,x_k) \rightarrow 0,
\]
kui $m,k\rightarrow 0$.
Seega koordinaatide jadad $(\xi_i^m)_{m=1}^\infty$, $i=1,\dots,n$ on Cauchy arvjadad, mis koonduvad, s.t. $\xi_i^m\rightarrow\xi_i$ protsessis $m\rightarrow\infty$.
Ruumides $m_n$ ja $l_p^n$ on koondumine samaväärne koordinaatide koondumisega, mistõttu 
\[
	x_m = (\xi_1^m,\dots,x_n^m) \rightarrow (\xi_1,\dots \xi_n)
\]
protsessis $m\rightarrow \infty$.

\begin{yl}
	Näidata, et ruum $c_0$ on kinnine ruumis $m$.
\end{yl}

\subsection{Teoreem üksteisesse sisestatud keradest}

\begin{theorem}[Teoreem üksteisesse sisestatud keradest]
	Meetriline ruum on täielik parajasti siis, kui temas üksteisesse sisestatud kinniste kerade jadas kerade raadiuste nullile lähenemisest järeldub, et neil keradel on ühine punkt.
\end{theorem}
\begin{toestus}
	\begin{enumerate}
		\item[]
		\item Eeldame, et täielikus meetrilises ruumis on antud kinniste kerade jada $(B_n)$, kus $B_n = \overline{B}(a_n,r_n)$ nii, et $B_1\supset B_2\supset \dots$ ja $r_n\rightarrow 0$.
		Näitame kõigepealt, et keskpunktide jada $a_n$ on fundamentaaljada.
		Olgu $\epsilon > 0$ suvaline ja $N$ selline, et iga $n>N$ korral $r_n < \epsilon$.
		Kui nüüd $n,m>N$ ja näiteks $m\geq n$, siis $a_m\in B_m\subset B_n = \overline{B}(a_n,r_n)$, mistõttu $d(a_m,a_n)\leq r_n<\epsilon$, kui $m,n > N$.
		Oleme näidanud, et $(a_n)$ on Cauchy jada.

		Ruumi täielikkuse tõttu $a_n\underset{n}{\rightarrow} a$.
		Näitame, et $a\in B_n$ iga $n=1,2,\dots$ korral.
		Valime vabalt $B_n$. 
		Kuna $a_n,a_{n+1},\dots\in B_n$, $a_{n+k}\underset{k}{\rightarrow}a$ ja $B_n$ on kinnine, siis $a\in B_n$.
		\item Olgu meetriline ruum ning selles suvaline Cauchy jada $(x_n)$.
		Peame näitama, et see ruum on täielik ehk et jada $(x_n)$ koondub.
		Eraldame selleks jadast $(x_n)$ mingi koonduva osajada.

		Jada $(x_n)$ fundamentaalsuse tõttu leidub selline liige $x_{n_1}$, et iga $n\geq n_1$ korral 
		\[
			d(x_n,x_{n_1})<\frac{1}{2}.
		\]
		Kui on valitud liikmed $x_{n_1},\dots,x_{n_{k-1}}$, kusjuures $n_1<n_2<\dots<n_{k-1}$, siis leiame $x_{n_k}$ nii, et kõikide $n\geq n_k$ korral $n_k > n_{k-1}$ ja $d(x_n,x_{n_k}) < \frac{1}{2^k}$.

		Vaatleme kinniste kerade jada $(B_k)=\overline{B}(x_{n_k},\frac{1}{2^{k-1}})$, $k=1,2,\dots$ ning näitame, et need on üksteisesse sisestatud.
		Kui $x\in B_{k+1}$ ehk $d(x,x_{n_{k+1}})\leq \frac{1}{2^k}$, siis
		\begin{align*}
			d(x,x_{n_k}) &\leq d(x,x_{n_{k+1}}) + d(x_{n_{k+1}},x_{n_k}) \\
			&< \frac{1}{2^k} + \frac{1}{2^k} = \frac{1}{2^{k-1}},
		\end{align*}
		mistõttu $x\in B_k$.

		Et kerade $B_k$ raadiused $\frac{1}{2^{k-1}}$ lähenevad nullile, siis on neil keradel ühine punkt $x$.
		Seejuures $x_{n_k}\underset{k}{\rightarrow}x$, sest
		\[
			d(x_{n_k},x)\leq \frac{1}{2^{k-1}} \underset{k}{\rightarrow} 0.
		\]
		\qed
	\end{enumerate}
\end{toestus}

\begin{yl}
	Näidata, et kerade ühine punkt $a$ on ainus.
\end{yl}

\subsection{Baire'i teoreem}

\begin{theorem}[Baire'i teoreem]
	Kui täielik meetriline ruum avaldub loenduva hulga kinniste hulkade ühendina, siis vähemalt üks neist hulkadest sisaldab mingit kera.
\end{theorem}
\begin{toestus}
	Avaldugu täielik meetriline ruum $X$ ühendina $X=\bigcup_{n=1}^\infty F_n$, kus hulgad $F_n$ on kinnised. Oletame vastuväiteliselt, et ükski hulkadest $F_n$ ei sisalda ühtegi kera.

	Olgu positiivne arvjada $\epsilon_n \rightarrow 0$. Kuna $X$ sisaldab kerasid, siis $F_1\neq X$ ehk $X\setminus F_1\neq \emptyset$. Hulk $X\setminus F_1$ on lahtine, järelikult sisaldab ta mingit kinnist kera $B_1 = \overline{B}(x_1,r_1)$, kusjuures võime eeldada, et $r_1 < \epsilon_1$.

	Hulk $F_2$ ei sisalda kerasid, mistõttu ei sisalda ta ka kera $B(x_1,r_1)$. Kui $(X\setminus F_2)\cap B(x_1,r_1) = \emptyset$, siis $B(x_1,r_1)\subset X\setminus(X\setminus F_2) = F_2$. Seega $(X\setminus F_2)\cap B(x_1,r_1)\neq \emptyset$. Kuna hulk $(X\setminus F_2)\cap B(x_1,r_1)$ on lahtine, siis sisaldab ta mingit kinnist kera $B_2 = \overline{B}(x_2,r_2), r_2 < \epsilon_2$.

	Analoogiliselt jätkates saame, et $(X\setminus F_n)\cap B(x_{n-1},r_{n-1})$ sisaldab mingit kinnist kera $B_n = \overline{B}(x_n,r_n),r_n < \epsilon_n$, ning niiviisi moodustub üksteisesse sisestatud kerade jada $B_1\supset \dots\supset B_n\supset \dots$, kusjuures $B_n\in X\setminus F_n$ ja kerade raadiused lähenevad nullile. Eelmise teoreemi põhjal on keradel $B_n$ ühine punkt $a$. Seega $\forall n\in\N: a\in B_n\subset X\setminus F_n$, millest järeldub, et $a\notin F_n$ ühegi $n$ korral. See aga tähendab, et $a\notin \bigcup_{n=1}^\infty F_n$, mis läheb vastuollu võrdusega $X=\bigcup_{n=1}^\infty F_n$.
	\qed
\end{toestus}

\section{Meetrilise ruumi täield}

\begin{definition}
	Meetrilisi ruume $X_0$ ja $X_1$ vastavalt kaugustega $d_0$ ja $d_1$ nimetatakse isomeetrilisteks, kui leidub bijektsioon $\varphi : X_0\rightarrow X_1$, mis säilitab elementide vahelise kauguse, s.t
	\[
		\forall x,y\in X_0: \h d_1(\varphi(x),\varphi(y)) = d_0(x,y).
	\]
	Bijektsiooni $\varphi$ nimetatakse isomeetriaks.
\end{definition}

\begin{definition}
	Meetrilise ruumi $X_0$ täieldiks nimetatakse niisugust meetrilist ruumi $X$, mille puhul $X_0$ on isomeetriline ruumi $X$ mingi kõikjal tiheda alamruumiga.
\end{definition}

Näiteks on $\Q$ täieldiks $\R$.

\begin{lause}
	Täieliku meetrilise ruumi alamruumi täieldiks on tema sulund.
\end{lause}

\begin{toestus}
	Olgu $X_0$ täieliku meetrilise ruumi alamruum. 
	Tema sulund $\overline{X_0}$, olles täieliku meetrilise ruumi kinnine osahulk, on samuti täielik meetriline ruum, milles $X_0$ on kõikjal tihe.
	Vajalikuks isomeetriaks on ühikteisendus $I:X_0\rightarrow X_0$, $I(x) = x$, $x\in X_0$.
\end{toestus}

\begin{theorem}
	Igal meetrilisel ruumil leidub täield, mis on isomeetria täpsuseni üheselt määratud.
\end{theorem}
\begin{toestus}
	Olgu $X_0$ meeriline ruum kaugusega $d_0$. Konstrueerime ruumile $X_0$ täieldi.

	Vaatleme ruumi $X_0$ elementidest moodustatud Cauchy jadade hulka.
	Cauchy jadad $(x_n)$ ja $(y_n)$ loeme ekvivalentseteks, kui $d_0(x_n,y_n)\rightarrow 0$. Tegemist on ekvivalentsusseosega, sest
	\begin{enumerate}
		\item Cauchy jada $(x_n)$ on iseendaga ekvivalentne;\hfill (refleksiivsus)
		\item kui $(x_n)$ ja $(y_n)$ on ekvivalentsed, siis kauguse $d_0$ sümmeetria tõttu on ka $(y_n)$ ja $(x_n)$ ekvivalentsed;\hfill (sümmeerilisus)
		\item Kui $(x_n)$ ja $(y_n)$ ning $(y_n)$ ja $(z_n)$ on ekvivalentsed, siis kolmnurga võrratusest
		\[
			d_0(x_n,z_n) \leq d_0(x_n,y_n) + d_0(y_n,z_n)
		\]
		järeldub, et $d_0(x_n,z_n)\rightarrow 0$, s.t. ka $(x_n)$ ja $(z_n)$ on ekvivalentsed.\hfill (transitiivsus)
	\end{enumerate}

	Seega tekib kõigi ruumi $X_0$ elementidest moodustaud Cauchy jadade hulgas klassijaotus. Loeme hulga $X$ elementideks omavahel ekvivalentsete Cauchy jadade klassid, s.t. $X$ on faktorhulk vaadeldava seose järgi.

	Järgnevalt muudame hulga $X$ meetriliseks ruumiks. Kui $x,y\in X$, siis valime suvalised Cauchy jadad $(x_n)\in x$ ja $(y_n)\in y$ ning defineerime
	\[
		d(x,y) = \lim_n d_0(x_n,y_n).
	\]
	Veendume selle definitsiooni korrektsuses ehk et igale elemendipaarile $x,y\in X$ seatakse vastavusse kindel arv $d(x,y)$.

	Nelinurga võrratusest
	\[
		|d_0(x_n,y_n) - d_0(x_m,y_m)| \leq d_0(x_n,x_m) + d_0(y_n,y_m)
	\]
	järeldub (???), et jada $d_0(x_n,y_n)$ on Cauchy arvjada. Järelikult see koondub ja piirväärtus eksisteerib. Kui valiksime klassidest $x$ ja $y$ teised esindajad $(x_n')\in x$ ja $(y_n')\in y$, siis nelinurga võrratusest
	\[
		|d_0(x_n',y_n') - d_0(x_n,y_n)| \leq d_0(x_n',x_n) + d_0(y_n',y_n)
	\]
	saaksime, et
	\[
		\lim_n d_0(x_n',y_n') = \lim_n d_0(x_n,y_n).
	\]
	Seega on $d$ definitsioon korrektne.

	Näitame, et $d$ rahuldab meetrika aksioome.
	\begin{enumerate}
		\item Kui $x=y$, siis $d(x,y)$ arvutamiseks vajaminevad Cauchy jadad on ekvivalentsed, s.t $d(x,y) = 0$.
		Vastupidi, kui $d(x,y)=0$, siis Cauchy jadad $(x_n)\in x$ ja $(y_n)\in y$ on ekvivalentsed, mistõttu peavad nad kuuluma samasse klassi, seega $x=y$.\hfill (identsus)
		\item 
		\[
			d(x,y) = \lim_n d_0(x_n,y_n) = \lim_n d_0(y_n,x_n) = d(y,x).
		\]
		\hfill (sümmeetria)
		\item Olgu $(x_n)\in x$, $(y_n)\in y$ ja $(z_n)\in z$, siis võrratus
		\[
			d_0(x_n,y_n) \leq d_0(x_n,z_n) + d_0(z_n,y_n)
		\]
		annab piiril $n\rightarrow 0$ kolmnurga võrratuse kauguse $d$ jaoks.\hfill (kolmnurga võrratus)
	\end{enumerate}
	
	Vaatleme ruumi $X$ osahulka $X_1$, mis koosneb sellistest ekvivalentsiklassidest, mis sisaldavad mingit ruumi $X_0$ elementide konstantset jada.
	Defineerime kujutuse $\varphi:X_0\rightarrow X_1$, seades elemendile $x_0\in X_0$ vastavusse konstantset jada $(x_0,x_0,\dots)$ sisaldava ekvivalentsiklassi.
	Ruumi $X_1$ ja kujutuse $\varphi$ definitsioonidest on selge, et $\varphi$ on sürjektsioon. Olgu $x_0,y_0\in X_0$, siis $(x_0,x_0,\dots)\in\varphi(x_0)$ ja $(y_0,y_0,\dots)\in\varphi(y_0)$, mistõttu
	\[
		d(\varphi(x_0),\varphi(y_0)) = \lim_n d_0(x_0,y_0) = d_0(x_0,y_0).
	\]
	Sellega oleme näidanud, et $\varphi: X_0\rightarrow X_1$ on isomeetria.

	Tõestame alamruumi $X_1$ kõikjal tiheduse ruumis $X$. Valime vabalt $x\in X$ ja $\epsilon > 0$. Olgu $(x_n)\in x$. Siis leidub $N\in\N$ nii, et $d_0(x_n,x_m) < \frac{\epsilon}{2}$, kui $n,m \geq N$. Olgu $y$ ekvivalentsiklass, mis sisaldab konstantset jada $(x_N,x_N,\dots)$. Siis $y\in X_1$ ja
	\[
		d(x,y) = \lim_n d_0(x_n,x_N) \leq \frac{\epsilon}{2} < \epsilon.
	\]
	Oleme tõestanud, et $\overline{X_1} = X$.

	Veendume ruumi $X$ täielikkuses.
	Olgu $x^{(n)}\in X$ Cauchy jada. Ruumi $X_1$ kõikjal tiheduse tõttu saame leida elemendid $y^{(n)}\in X_1$ nii, et
	\[
		d(y^{(n)}, x^{(n)}) < \frac{1}{n},n\in\N.
	\]
	Iga klass $y^{(n)}$ sisaldab mingi konstantse jada $(y_n,y_n,\dots)$, kus $y_n\in X_0$. Seejuures jada $(y_0)$ ise on Cauchy jada ruumis $X_0$, sest $n,m\rightarrow\infty$ korral
	\begin{align*}
		d_0(y_n,y_m) &= d(y^{(n)}, y^{(m)}) \\
		&\leq d(y^{(n)}, x^{(n)}) + d(x^{(n)}, x^{(m)}) + d(x^{(m)}, y^{(m)}) \\
		&\leq d(x^{(n)}, x^{(m)}) + \frac{1}{n} + \frac{1}{m} \rightarrow 0.
	\end{align*}
	Niisiis leidub $x\in X$, kuhu kuulub jada $(y_n)$.
	Seejuures
	\begin{align*}
		d(x^{(n)},x) &\leq d(x^{(n)}, y^{(n)}) + d(y^{(n)},x) \\
		&\leq \frac{1}{n} + \lim_{m\rightarrow \infty} d_0(y_n,y_m).
	\end{align*}
	\dots

\end{toestus}

\section{Pidevad operaatorid meetrilistes ruumides}

\begin{definition}
	Operaatorit $f:X\rightarrow Y$ nimetatase pidevaks punktis $a\in X$, kui
	\[
		\forall \epsilon > 0 \h\exists \delta > 0 : d(x,a)<\delta \Rightarrow d(f(x),f(a))<\epsilon.
	\]
	Operaatorit $f$ nimetatakse pidevaks, kui ta on pidev ruumi $X$ igas punktis.
\end{definition}

\begin{theorem}
	Operaator $f:X\rightarrow Y$ on pidev punktis $a\in X$ parajasti siis, kui mistahes koondva jada $x_n\rightarrow a$ korral $f(x_n)\rightarrow f(a)$.
\end{theorem}

\begin{toestus}
	\begin{enumerate}
		\item[]
		\item Olgu operaator $f$ pidev punktis $a$. 
		Vaatleme koonduvat jada $x_n\rightarrow a$.
		Koonduvuse tõttu leidub $N$ nii, et kui $n > N$, siis $d(x_n,a)<\delta$.
		Kuid siis $d(f(x_n),f(a))<\epsilon$, mistõttu $f(x_n)\rightarrow f(a)$.
		\item Eeldame, et iga jada $x_n\rightarrow a$ korral $f(x_n)\rightarrow f(a)$.
		Oletame vastuväiteliselt, et operaator $f$ ei ole pidev punktis $a$. Siis leidub $\epsilon > 0$ nii, et iga $\delta > 0$ korral leidub $x_\delta$, mille puhul $d(x_\delta, a)<\delta$, kuid $d(f(x_\delta), f(a))\geq \epsilon$. 
		Valime jada $\delta_n\rightarrow 0$, näiteks $\delta_n = \frac{1}{n}$ ning vastavad elemendid $x_n$, mille korral $d(x_n,a)<\delta_n$, aga $d(f(x_n),f(a))\geq\epsilon$.
		Siis $x_n\rightarrow a$, aga $f(x_n)\not\rightarrow f(a)$.
	\end{enumerate}
	\qed
\end{toestus}

\begin{theorem}
	Operaator on pidev parajasti siis, kui lahtiste hulkade originaalid (selle operaatori järgi) on lahtised.
\end{theorem}

\begin{toestus}
	\begin{enumerate}
		\item[]
		\item Olgu $f:X\rightarrow Y$ pidev ja olgu hulk $G\subset Y$ lahtine. 
		Vaatleme suvalist elementi $a\in f^{-1}(G)$.
		Siis $f(a)\in G$ ja hulga $G$ lahtisuse tõttu leidub kera $B(f(a),\epsilon)\subset G$.
		Operaatori $f$ pidevuse tõttu leidub kera $B(a,\delta)$ nii, et $B(a,\delta)\subset f^{-1}(B(f(a), \epsilon))$.
		Sellepärast $B(a,\delta)\subset f^{-1}(G)$, s.t. element $a$ sisaldub hulgas $f^{-1}(G)$ koos teda ümbritseva keraga.
		Seega $f^{-1}(G)$ on lahtine.
		\item Eeldame, et $f^{-1}(G)$ on lahtine alati, kui $G\subset Y$ on lahtine.
		Valime vabalt $a\in X$ ja $\epsilon > 0$.
		Kuna $B(f(a), \epsilon)\subset Y$ on lahtine, siis $f^{-1}(B(f(a), \epsilon))$ on lahtine.
		Järelikult leidub punktil $a\in f^{-1}(B(f(a), \epsilon))$ ümbrus $B(a,\delta)$ nii, et $B(a,\delta)\subset f^{-1}(B(f(a), \epsilon))$, mis tähendab operaatori $f$ pidevust punktis $a$. Kuna $a\in X$ oli suvaline, siis $f$ on pidev.
	\end{enumerate}
\end{toestus}

\begin{theorem}
	Operaator on pidev parajasti siis, kui kinniste hulkade originaalid (selle operaatori järgi) on kinnised.
\end{theorem}

\begin{toestus}
	Vaadeldes kõiki kinniseid hulki $F\subset Y$, paneme tähele, et hulgad $Y\setminus F$ esindavad parajasti kõiki lahtiseid hulki ruumis $Y$. 
	Hulk $f^{-1}(F)\subset X$ on kinnine parajasti siis, kui hulk $X\setminus f^{-1}(F) = f^{-1}(Y\setminus F)$ on lahtine.
	Eelmise teoreemi põhjal on aga $f$ pidev parajasti siis, kui hulga $Y\setminus F$ lahtisusest järeldub $f^{-1}(Y\setminus F)$ lahtisus.
	\qed
\end{toestus}

\begin{theorem}
	Järgmised tingimused on samaväärsed:
	\begin{enumerate}
		\item operaator $f:X\rightarrow Y$ on pidev;
		\item iga $A\subset X$ korral $f(\overline{A})\subset \overline{f(A)}$;
		\item iga $B\subset Y$ korral $\overline{f^{-1}(B)}\subset f^{-1}(\overline{B})$;
		\item iga $B\subset Y$ korral $f^{-1}(B^\circ)\subset (f^{-1}(B))^\circ$.
	\end{enumerate}
\end{theorem}

\begin{toestus}
	$(1^\circ\Rightarrow 2^\circ)$
	Eeldame, et $f$ on pidev.
	Sisalduvusest $f(A)\subset\overline{f(A)}$ järeldub, et 
	$$
		f^{-1}(f(A))\subset f^{-1}(\overline{f(A)})
	$$
	ning $A\subset f^{-1}(f(A))$ tõttu ka 
	$$
		A\subset f^{-1}(\overline{f(A)}).
	$$
	Et aga $\overline{f(A)}$ kinnisuse ja $f$ pidevuse tõttu $f^{-1}(\overline{f(A)})$ on kinnine, siis $\overline{A}\subset f^{-1}(\overline{f(A)})$. Viimasest järeldub, et
	\[
		f(\overline{A})\subset f(f^{-1}(\overline{f(A)})) \subset \overline{f(A)}.
	\]

	$(2^\circ\Rightarrow 3^\circ)$
	Valime vabalt $B\subset Y$ ning rakendame hulgale $A=f^{-1}(B)$ tingimust $2^\circ$.
	Siis
	\[
		f(\overline{f^{-1}(B)}) \subset \overline{f(f^{-1}(B))} \subset \overline{B},
	\]
	mille abil saame
	\[
		\overline{f^{-1}(B)} \subset f^{-1}(f(\overline{f^{-1}(B)}))\subset f^{-1}(\overline{B}).
	\]

	$(3^\circ\Rightarrow 4^\circ)$
	Olgu $B\subset Y$, siis tingimuse $3^\circ$ põhjal
	\[
		\overline{f^{-1}(Y\setminus B)} \subset f^{-1}(\overline{Y\setminus B}).
	\]
	Arvestades võrdust $B^\circ = Y\setminus \overline{Y\setminus B}$, saame nüüd
	\begin{align*}
		f^{-1}(B^\circ) &= f^{-1}(Y\setminus \overline{Y\setminus B}) = X\setminus f^{-1}(\overline{Y\setminus B}) \\
		&\subset X\setminus \overline{f^{-1}(Y\setminus B)} = X\setminus \overline{X \setminus f^{-1}(B)} = (f^{-1}(B))^\circ.
	\end{align*}

	$(4^\circ\Rightarrow 1^\circ)$
	Olgu $G\subset Y$ lahtine, siis $G = G^\circ$.
	Tingimuse $4^\circ$ põhjal
	\[
		f^{-1}(G) = f^{-1}(G^\circ) \subset (f^{-1}(G))^\circ,
	\]
	mistõttu $f^{-1}(G)$ on lahtine.
	Seega on $f$ pidev operaator.
	\qed
\end{toestus}

\section{Banachi püsipunkti printsiip}

\begin{definition}
	Operaatori $f:X\rightarrow X$ püsipunktiks nimetatakse sellist elementi $x^*\in X$, et $f(x^*) = x^*$.
\end{definition}

Operaatori $f$ püsipunktid on parajasti võrrandi $x=f(x)$ võrrandid.

\begin{definition}
	Öeldakse, et operaator $f:X\rightarrow X$ on ahendav (ehk lähenemisoperaator ehk suruv operaator), kui leidub $q < 1$ nii, et
	\[
		\forall x,y\in X:\h d(f(x),f(y)) \leq qd(x,y).
	\]
\end{definition}

\begin{definition}
	Lipschitzi tingimust rahuldavad operaatorid $f$ on sellised, mille korral leidub arv $L$ nii, et
	\[
		\forall x,y\in X:\h d(f(x),f(y)) \leq Ld(x,y).
	\]	
\end{definition}

Kui $x_n\rightarrow a$ ehk $d(x_n,a)\rightarrow 0$, siis sändvitš teoreemi tõttu ka $f(x_n)\rightarrow f(a)$.

\begin{theorem}[Banachi püsipunkti printsiip]
	Kui $X$ on täielik meetriline ruum, siis ahendaval operaatoril $f:X\rightarrow X$ on parajasti üks püsipunkt.
\end{theorem}

\begin{toestus}
	Valime vabalt $x_0\in X$.
	Olgu $x_1 = f(x_0), x_2 = f(x_1), \dots, x_n = f(x_{n-1}),\dots$.
	Näitame, et jada $x_n$ on Cauchy jada.
	Kuna
	\begin{align*}
		d(x_n,x_{n+1}) &= d(f(x_{n-1}), f(x_n)) \leq qd(x_{n-1},x_n) \\
		&\leq q^2 d(x_{n-2}, x_{n-1}) \leq \dots \leq q^n d(x_0,x_1),
	\end{align*}
	siis protsessis $n\rightarrow 0$
	\begin{align*}
		d(x_n,x_{n+p}) &\leq d(x_n,x_{n+1}) + \dots + d(x_{n+p-1}, x_{n+p}) \\
		&\leq q^n d(x_0,x_1) + \dots + q^{n+p-1}d(x_0,x_1) \\
		&\leq \sum_{k=0}^\infty q^{n+k} d(x_0,x_1) = \frac{q^n}{1-q} d(x_0,x_1) \rightarrow 0.
	\end{align*}

	Seega $\forall \epsilon > 0\h\exists N\in \N:\h d(x_n,x_{n+p})<\epsilon$ iga $p = 1,2,\dots$ korral ehk $(x_n)$ on Cauchy jada.
	Ruumi $X$ täielikkuse tõttu jada $(x_n)$ koondub, olgu $x_n\rightarrow x^*$.
	Operaatori $f$ pidevusest järeldub nüüd, et
	\[
		f(x^*) = \lim_n f(x_n) = \lim_n x_{n+1} = x^*,
	\]
	s.t. $x^*$ on operaatori $f$ püsipunkt.

	Kui lisaks püsipunktile $x^* = f(x^*)$ leiduks veel $x^{**} = f(x^{**})$, siis
	\[
		d(x^*, x^{**}) = d(f(x^*), f(x^{**})) \leq qd(x^*, x^{**}).
	\]
	Tingimuse $q < 1$ tõttu kehtib viimane parajasti siis, kui $d(x^*, x^{**}) = 0$, mistõttu $x^* = x^{**}$.
	\qed
\end{toestus}

\begin{theorem}
	Kui $X$ on täielik meetriline ruum ja leidub $n\in \N$ selliselt, et $f^n := \underbrace{f\circ\dots\circ f}_n$ on ahendav, siis operaatoril $f$ on parajasti üks püsipunkt.
\end{theorem}

\begin{toestus}
	Banachi püsipunkti printsiibi põhjal leidub operaatoril $f^n$ parajasti üks püsipunkt $x^* = f^n(x^*)$.
	Kuna $f(x^*) = f(f^n(x^*)) = f^n(f(x^*))$, siis ka $f(x^*)$ on operaatori $f^n$ püsipunkt. Seepärast $f(x^*) = x^*$.

	Kui veel $x^{**} = f(x^{**})$, siis $x^{**} = f^n(x^{**})$, seega $x^{**} = x^*$.
\end{toestus}

\section{Hulga kompaktsus ja tõkestatus}

\subsection{Suhteline kompaktsus ja kompaktsus}

\begin{definition}
	Hulka $K$ meetrilises ruumis nimetatakse suhteliselt kompaktseks ruumiks, kui igast $K$ elementidest moodustatud jadast saab eraldada koonduva osajada.
\end{definition}

\begin{definition}
	Hulka $K$ nimetatakse kompaktseks, kui see on suhteliselt kompaktne ja kinnine.
\end{definition}

Näiteks on Bolzano-Weierstrassi teoreemi põhjal iga tõkestatud hulk ruumis $\R$ suhteliselt kompaktne ja iga tõkestatud kinnine hulk kompaktne. Samuti on iga lõplik hulk kompaktne, sest tema elementidest moodustatud jadas vähemalt üks element esineb lõpmatu arv kordi. Selle elemendi abil saab jadast eraldada konstantse osajada. Vastunäiteks, $\N$ ei ole suhteliselt kompaktne ruumis $\R$.

\begin{lause}
	Suhteliselt kompaktse hulga sulund on kompaktne.
\end{lause}

\begin{toestus}
	Et hulga sulund on alati kinnine, siis piisab tõestada, et suhteliselt kompaktse hulga $K$ sulund $\overline{K}$ on suhteliselt kompaktne.
	Vaatleme suvalist jada $x_n\in \overline{K}$.
	Valime $y_n\in K,n\in\N$ nii, et $d(x_n,y_n)\rightarrow 0$.
	Hulga $K$ suhtelise kompaktsuse tõttu võime leida $y_n$ osajada nii, et $y_n\xrightarrow[n\in N']{} x$, kus $N'\subset\N$.
	Siis saame võrratusest
	\[
		d(x_n,x) \leq d(x_n,y_n) + d(y_n,x),
	\]
	et $x_n\xrightarrow[n\in N']{} x$.
\end{toestus}

\begin{lause}
	Kompaktne meetriline ruum on täielik.
\end{lause}

\begin{toestus}
	Eelduse kohaselt saab igast Cauchy jadast eraldada koonduva osajada.
	Kui aga Cauchy jada mingi osajadak koondub, siis koondub ka jada ise.
\end{toestus}

\subsection{Suhteliselt kompaktse hulga tõkestatus}

\begin{definition}
	Hulka meetrilises ruumis nimetatakse tõkestatuks, kui ta sisaldab mingit kera.
\end{definition}

\begin{lause}
	Iga suhteliselt kompaktne hulk on tõkestatud.
\end{lause}

\begin{toestus}
	Oletame vastuväiteliselt, et suhteliselt kompaktne hulk $K$ ruumis $X$ ei ole tõkestatud. 
	Valime suvalise elemendi $a\in X$. Kuna $K\not\subset B(a,1)$, siis leidub $x_1\in K$ nii, et $d(x_1,a)\geq 1$. 
	Kuna $K\not\subset B(a,2)$, siis leidub $x_2\in K$ nii, et $d(x_2,a)\geq 2$. 
	Niiviisi jätkates leiame $x_n\in K$ nii, et $d(x_n,a)\geq n$, $n\in\N$, mistõttu $d(x_n,a)\rightarrow \infty$ protsessis $n\rightarrow \infty$. 
	Hulga suhteline kompaktsus lubab eraldada jadast $(x_n)$ koonduva osajada $x_{n_k}\rightarrow x\in X$. 
	Kauguse pidevuse tõttu $d(x_{n_k}, a)\rightarrow d(x,a)$ protsessis $k\rightarrow \infty$, mis on vastuolus tingimusega $d(x_{n_k},a)\rightarrow\infty$.
	\qed
\end{toestus}

\subsection{Hausdorfi teoreem}

\begin{definition}
	Olgu $X$ meetriline ruum ja $\epsilon>0$.
	Hulga $A\subset X$ $\epsilon$-võrguks nimetatakse hulka $B\subset X$, mille puhul
	\[
		\forall x\in A\h\exists y\in B:\h d(x,y)<\epsilon.
	\]
\end{definition}

Näiteks, kui $\epsilon > \frac{1}{2}$, siis täisarvude hulk $\Z$ on $\epsilon$-võrguks ruumile $\R$ ja igale osahulgale $A\subset \R$, sest vahemikud $B(\kappa,\epsilon),\kappa\in\Z$, katavad ruumi $\R$.

\begin{theorem}[Hausdorfi teoreem]
	Meetrilise ruumi $X$ osahulga $K$ suhteliseks kompaktsuseks on tarvilik (ning ruumi $X$ täielikkuse korral ka piisav), et iga $\epsilon > 0$ korral leiduks hulgale $K$ lõplik $\epsilon$-võrk.
\end{theorem}
\begin{toestus}
	\begin{enumerate}
		\item[]
		\item Olgu hulk $K$ suhteliselt kompaktne.
	\end{enumerate}
\end{toestus}

\subsection{Tõkestatus normeeritud ruumis}

\section{Pidevad funktsionaalid kompaktsetel hulkadel (Weierstrassi teoreemid)}



\section{Arzelà-Ascoli teoreem}

\subsection{Tarvilikkuse tõestus}

\section{Lineaarsete operaatorite pidevus ja tõkestatus}

Olgu $X$ ja $Y$ vektorruumid üle ühe ja sama korpuse $\K$.

\begin{definition}
	Operaatorit $A:X\rightarrow Y$ nimetatakse lineaarseks, kui
	\begin{enumerate}
		\item $A(x_1+x_2) = A(x_1) + A(x_2)$, kus $x_1,x_2\in X$, \hfill (aditiivsus)
		\item $A(\lambda x) = \lambda A(x)$, kus $x\in X$ ja $\lambda \in \K$.\hfill (homogeensus)
	\end{enumerate}
\end{definition}

\section{Põhimõisted}

\begin{definition}
	Hulk $X$ koos kujutusega $d:X\times X\rightarrow \R$ on meetriline ruum, kui $\forall x,y,z\in X$ korral:
	\begin{enumerate}
		\item $d(x,y)=0 \Leftrightarrow x=y$; \hfill (identsus)
		\item $d(x,y) = d(y,x)$; \hfill (sümmeetria)
		\item $d(x,y) + d(y,z) \geq d(x,z)$. \hfill (kolmnurga võrratus)
	\end{enumerate}
\end{definition}

\begin{definition}
	Hulk $A\subset X$ on kõikjal tihe ruumis $X$, kui $\overline{A} = X$.
\end{definition}

\begin{definition}
	Meetriline hulk on separaabel, kui temas leidub kõikjal tihe ülimalt loenduv hulk.
\end{definition}

\begin{definition}
	Meetrilise ruumi elementide jada $x_n$ nimetatakse Cauchy jadaks ehk fundamentaaljadaks, kui
	\[
		\forall \epsilon > 0\h\exists N\in\N:\h n,m>N\Rightarrow d(x_n,x_m)<\epsilon.
	\]
\end{definition}

\begin{definition}
	Meetrilist ruumi nimetatakse täielikuks, kui temas iga Cauchy jada koondub.
\end{definition}

\begin{definition}
	Meetrilise ruumi $X_0$ täieldiks nimetatakse niisugust meetrilist ruumi $X$, mille puhul $X_0$ on isomeetriline ruumi $X$ mingi kõikjal tiheda alamruumiga.
\end{definition}

\end{document}