\documentclass{article}[12pt]
\usepackage[T1]{fontenc}
\usepackage[utf8]{inputenc}
\usepackage[estonian]{babel}
\usepackage{amsmath}
\usepackage{amssymb}
\usepackage{amsthm}
\usepackage{latexsym}
\usepackage{mathtools}
\usepackage{setspace}
\usepackage{ifthenx}
\usepackage{graphicx}
\usepackage{enumitem}
\usepackage{import}
\usepackage{pdfpages}
\usepackage{transparent}
\usepackage{xifthen}
\usepackage{float}
\usepackage[a4paper, lmargin=0.1666\paperwidth, rmargin=0.1666\paperwidth, tmargin=0.1111\paperheight, bmargin=0.1111\paperheight]{geometry} %margins

% formatting section headings
\usepackage{titlesec}
\titleformat{\section}[block]{\Large\bfseries\filcenter}{\newpage\thesection}{1em}{}[\vspace{2ex}]
\titleformat{\subsection}{\large\bfseries\filcenter}{\thesubsection}{1em}{}[\vspace{1ex}]


\linespread{1.3}

% spacing
\newcommand{\h}{\hspace{0.2cm}}
% set symbols
\newcommand{\R}{\mathbb{R}}
\newcommand{\N}{\mathbb{N}}
\newcommand{\Q}{\mathbb{Q}}
\newcommand{\Z}{\mathbb{Z}}
\newcommand{\K}{\mathbb{K}}
\newcommand{\C}{\mathbb{C}}

% norm
\newcommand{\norm}[1]{||#1||}

\newcommand{\toinf}{\rightarrow\infty}
\newcommand{\limtoinf}[2]{\forall \epsilon > 0 \h \exists N \in \mathbb{N}: k \geq N \Rightarrow \ifthenelse{#2=0}{\left\lvert #1\right\rvert}{\left\lvert #1-#2\right\rvert} <\epsilon}

% Ülesanne
\newtheorem{yl}{Ülesanne}[section]

% Definitsioon
\theoremstyle{definition}
\newtheorem{definition}{Definitsioon}[section]

% Teoreem
\theoremstyle{definition}
\newtheorem{theorem}{Teoreem}[section]

% Järeldus
\newtheorem{jareldus}{Järeldus}[section]

% Muuda default listi stiili
\setlist{noitemsep}
\setlist[1]{labelindent=\parindent}
\setlist[enumerate,1]{label = \arabic*$^\circ$, ref = \arabic*}
\setlist[enumerate,2]{label = \emph{\alph*}),ref 	= \theenumi.\emph{\alph*}}

\begin{document}

\title{Funktsionaalanalüüs 1}
\author{René Piik}
\date{2021 kevad}

\maketitle

\section*{Meta}

Õppejõud: Rainis Haller \newline
Hindamine:
\begin{enumerate}
	\item Kontrolltööd $2\times 20p = 40p$;
	\item Eksam $60p$;
	\item Lisaülesanded.
\end{enumerate}
Semestri lõpus saab ühe kontrolltöödest uuesti teha, kirja läheb sel juhul uuesti kirjutatud töö tulemus.

\tableofcontents

\section{Meetrilised ruumid}

Hulk $X$, mille puhul peame silmas kindlat kujutust $d:x\times X\rightarrow \R$, on meetriline ruum, kui $\forall x,y,z\in X$ korral:
\begin{enumerate}
	\item $d(x,y)=0 \Leftrightarrow x=y$;
	\item $d(x,y) = d(y,x)$;
	\item $d(x,y) + d(y,z) \geq d(x,z)$.
\end{enumerate}
Kusjuures viimasest saame tuletada tagurpidi kolmnurga võrratuse.
\begin{align*}
	d(x,y) + d(y,z) &\geq d(x,z) \\
	d(x,y) &\geq d(x,z) - d(y,z) \\
	d(x,y) &\geq \big\lvert d(x,z) - d(y,z) \big\rvert \h\h\text{(seda tuleb eraldi näidata)} \\
\end{align*} 
Saame alati mistahes hulga teisendada meetriliseks ruumiks kasutades kaugust
\[
	d(x,y) = \begin{cases}
		0,\h\h x=y, \\
		1,\h\h x\neq y
	\end{cases}
\]
Näited:
\begin{enumerate}
	\item $X=\R,\h d(x,y) = |x-y|$. Naturaalne kaugus;
	\item $X=\R,\h d(x,y) = |e^x-e^y|$;
	\item $X=\R,\h d(x,y) = |\arctan x - \arctan y|$. See kaugu on huvitav, sest maksimaalne kaugus kahe elemendi vahel on $\pi$. Ei esine lõpmatuid kaugusi.
	\item $X=\R,$
	\[
		d(x,y) = \begin{cases}
			|x-y|,\h\h |x-y| < 1, \\
			1,\h\h |x-y| \geq 1, \\
		\end{cases} = \min \{ |x-y|, 1 \};
	\]
	\item $X=\N,\h d(m,n) = \bigg\lvert \frac{1}{m} - \frac{1}{n} \bigg\rvert$. Seda kaugust kasutati Googles veebilehtede tähtsuste järjestamisel.
	\item Hammingu kaugus.
	Fikseerime $n\in \N$. 
	Olgu $X$ kõikide pikkusega $n$ binaarjärjendite hulk. 
	Hammingu kaugus $d(x,y)$ on vastavate erinevate bittide arv.
	\item Olgu $x=\{0,1\}$. Osutub, et kaugus $d$ iga $x,y\in X$ korral, kus
	\[
		d(x,y) = \min \{ k\in \N: x_k \neq y_k \},
	\]
	on kaugus. Samuti
	\[
		d(x,y) = \sum_{k=1}^\infty \frac{ |x_k-y_k| }{2^k}.
	\]
\end{enumerate}

\begin{yl}
	Olgu $X=\R, d(x,y) = |f(x)-f(y)|$. Leida tingimused, mida funktsioon $f$ peab rahuldama, et $d$ oleks kaugus hulgas $X$.	
\end{yl}

\subsection*{Koonduvus meetrilises ruumis}

\begin{definition}
	Öeldakse, et jada $x_n\in X$ koondub elemendiks $x\in X$, kui $\rho(x_n,x)\rightarrow 0$ protsessis $n\rightarrow \infty$.
	Teisisõnu
	\[
		\forall \epsilon > 0 \h\exists N\in \N : n\geq N\Rightarrow d(x_n,x)<\epsilon
	\]	
\end{definition}
\begin{definition}
	Jada $(x_n)$ nimetatakse statsionaarseks, kui leidub $N$ nii, et sellest kohast alates on kõik jada liikmed ($x_N,x_{N+1},\dots$) võrdsed.
\end{definition}
\begin{theorem}
	Statsionaarne jada koondub alati.
\end{theorem}

Järeldused:
\begin{enumerate}
	\item Koonduval jadal leidub täpselt üks piirelement;
	\item Koonduva jada iga osajada koondub samaks piirelemendiks;
	\item (kauguse pidevus) Kui meetrilises ruumis $(X,d)$ $x_n\rightarrow x$ ja $y_n \rightarrow y$, siis $d(x_n,y_n)\rightarrow d(x,y)$.
\end{enumerate}

\begin{yl}
	Tõesta, et diskreetses meetrilises ruumis koonduvad ainult statsionaarsed jadad.
\end{yl}

\section{Normeeritud ruum ja Banachi ruum}

\begin{definition}
	Vektorruumi $X$ nimetatakse normeeritud ruumiks, kui igale tema elemendile $x\in X$ on vastavusse seatud kindel reaalarv $||x||$, mida nimetatakse elemendi $x$ normiks, nii, et on täidetud tingimused:
	\begin{enumerate}
		\item $||x|| = 0 \Leftrightarrow x=0$ (samasuse aksioom);
		\item $||\lambda x|| = ||\lambda||\cdot ||x||$ (homogeensuse aksioom);
		\item $||x+y|| \leq ||x|| + ||y||$ (kolmnurga võrratus).
	\end{enumerate}	
\end{definition}

\begin{theorem}
	(Põhiteoreem) Iga normeeritud ruum (NR) on meetriline ruum (MR) kaugusega $\rho(x,y) = ||x-y||$.
\end{theorem}

\begin{jareldus}
	\begin{enumerate}
		\item $\forall x\in X: \norm{x} \geq 0$ ($\norm{x} = \norm{-x}$);
		\item $\bigg\lvert \norm{x} - \norm{y} \bigg\rvert \leq \norm{x+y}$;
		\item (skalaariga korrutamise pidevus) Kui $\alpha_n\rightarrow \alpha$ ($\K$) ja $x_n\rightarrow x$ ($X$), siis $\alpha_n x_n\rightarrow \alpha x$ ($X$);
		\item (liitmise pidevus) Kui $x_n\rightarrow x$ ja $y_n\rightarrow y$, siis $x_n+y_n \rightarrow x+n$.
	\end{enumerate}
\end{jareldus}

Täielikku normeeritud ruumi nimetatase Banachi ruumiks. (Mida tähendab täielik normeeritud ruum??)

\section{Klassikalised lõplikumõõtmelised normeeritud ruumid}

Vaatleme mittetriviaalset VR-i (üle $\K$) $X$. Olgu $d$ kaugus hulgas $X$.
Tarvilik tingimus: iga $M>0$ korral leiduvad punktid $x,y\in X$ nii, et $d(x,y) = M$.

Tarvilik ja piisav tingimus: $d$ on indutseeritud normi poolt parajasti siis, kui kehtivad kaks tingimust:
\begin{enumerate}
	\item $\forall x,y,z\in X :\h d(x,y) = d(x+z,y+z)$;
	\item $\forall \alpha\in \K \forall x,y\in X: \h d(x\alpha,y\alpha) = |\alpha|d(x,y)$.
\end{enumerate}
Sel juhul see norm on defineeritud seosega $\norm{x} = d(x,0)$.


Olgu $X=\{(\xi_1,\dots,\xi_n): \xi_k\in\K\}$ kõigi $n$-komponendiliste vektorite hulk.
Tähistame hulga $X$ elemente $x=(\xi_1,\dots,\xi_n),\h y=(\eta_1,\dots,\eta_n),\h z=(\zeta_1,\dots,\zeta_n)$.
Sellises hulgas on mitu võimalust kauguse defineerimiseks, vastavalt erinevad ka ruumide tähised.
\begin{enumerate}
	\item Ruumis $m_n$ defineeritakse kaugus
	\[
		\rho(x,y) = \max_{1\leq k \leq n} |\xi_k-\eta_k|.
	\]
	Aksioomide kontroll.
	\begin{enumerate}
		\item $\rho(x,y) = 0 \Leftrightarrow x=y$.
		Tarvilikus on ilmne. 
		Piisavus. 
		Olgu $\max_{1\leq k \leq n} |\xi_k-\eta_k| = 0$. 
		Siis peavad mõlema vektori kõik koordinaadid olema võrdsed, millest järeldub, et ka vektorid ise on võrdsed.
		\item $\rho(x,y) = \rho(y,x)$.
		Absoluutväärtuse märgi tõttu pole vahet, kumb punkt võetakse "esimeseks".
		\item $\rho(x,y) \leq \rho(x,z) + \rho(z,y)$.
		Iga indeksi korral
		\begin{align*}
			|\xi_k-\eta_k| &= |\xi_k-\zeta_k+\zeta_k-\eta_k| \\
			&\leq |\xi_k-\zeta_k|+|\zeta_k-\eta_k| \\
			&\leq \rho(x,z) + \rho(z,y).
		\end{align*}
		Leides maksimumi $k$ järgi, saamegi kolmnurga võrratuse.
	\end{enumerate}
	\item Olgu $1\leq p \leq \infty$.
	Ruumis $l_p^n$ defineeritakse kaugus
	\[
		\rho(x,y) = \left( \sum_{k=1}^n |\xi_k - \eta_k|^p \right)^{\frac{1}{p}}.
	\]
	Olulised erijuhud on $l_1^n$, kus $\rho(x,y) = \sum_{k=1}^n |\xi_k-\eta_k|$ ja $l_2^n$.
	Viimast ruumi tähistatakse ka $\R^n$ või $\C^n$, üldiselt $\K^n$.
	Ruum $\R^n$ on $n$-mõõtmeline eukleidiline ruum.

	Kauguse aksioomide $1^\circ$ ja $2^\circ$ kehtimist pole raske kontrollida.
	Aksioomi $3^\circ$ kehtivus järeldub näiteks ruumis $l_1^n$ absoluutväärtuse või mooduli omadustest, kuid kõrgema astme ruumide korral on see keerulisem.
	Kasutame üldjuhul viimase tingimuse tõestamiseks \textbf{Hölderi võrratust}.
\end{enumerate}

\begin{theorem}
	(Hölderi võrratus) Kui $a_k,b_k\in \K,\h k=1,\dots,n$ ja $1<p<\infty$ ning arv $q$ on määratud võrdusega
	\[
		\frac{1}{p} + \frac{1}{q} = 1,
	\]
	siis kehtib võrratus
	\[
		\sum_{k=1}^n |a_k b_k| \leq \left( \sum_{k=1}^n |a_k|^p \right)^{\frac{1}{p}} \cdot \left( \sum_{k=1}^n |b_k|^q \right)^{\frac{1}{q}}.
	\]
	Teistmoodi saaks kirjutada
	\[
		\norm{(a_1b_1,\dots,a_nb_n)} \leq \norm{(a_1,\dots,a_n)}_p \cdot \norm{(b_1,\dots,b_n)}_q.
	\]
\end{theorem}
Samuti peaks teadma \textbf{Minkowski võrratust}.
\begin{theorem}
	(Minkowski võrratus) Kui $a_k,b_k\in \K,\h k=1,\dots,n$ ja $1<p<\infty$, siis kehtib võrratus
	\[
		\left( \sum_{k=1}^n |a_k + b_k|^p \right)^{\frac{1}{p}} \leq \left( \sum_{k=1}^n |a_k|^p \right)^{\frac{1}{p}} + \left( \sum_{k=1}^n |b_k|^p \right)^{\frac{1}{p}}.
	\]
	\[
		\norm{(a_1,\dots,a_n) + (b_1,\dots,b_n)}_p \leq \norm{(a_1,\dots,a_n)}_p + \norm{(b_1,\dots,b_n)}_p.
	\]
\end{theorem}

\textit{Vaata moodlest lisamaterjalide alt Rainise lemmiktõestust Minkowski võrratusele!}

\begin{yl}
	Näidata, et
	\[
		\lim_{p\rightarrow \infty} \left( \sum_{k=1}^n |\xi_k-\eta_k|^p \right)^{\frac{1}{p}} = \max_{1\leq k\leq n} |\xi_k - \eta_k|.
	\]
	Selle ülesande tulemuse järgi on kaugus ruumis $m_n$ ruumide $l_p^n$ kauguste piirjuht, mistõttu on õigustatud tähistus $m_n = l_\infty^n$.
\end{yl}

\subsection*{Jadade koonduvus neis ruumides}

Osutub, et ruumis $(\K_n, \norm{\cdot}_p)$, kus $1\leq p\leq \infty$, $x_n\rightarrow x$\dots

\section{Klassikalised jadaruumid}

Siin vaatleme jadaruume, kus meetrilise ruumi $X$ elementideks on arvjadad $(\xi_k),\xi_k\in\K$.
Tähistame ruumi $X$ elemente $x=(\xi_k),y=(\eta_k),z=(\zeta_k)$.
\begin{enumerate}
	\item Tõkestatud jadade ruum $m$ on hulgana kirjeldatav:
	\[
		m=\{ (\xi_k):\xi_k\in\K,\sup_k |\xi_k|<\infty \},
	\]
	s.t. $x\in m$ parajasti siis, kui leidub arv $M$ (mis võib sõltuda elemendist $x$) nii, et $|\xi_k|\leq M$.
	Ruumis $m$ defineeritakse kaugus võrdusega
	\[
		\rho(x,y) = \sup_k |\xi_k-\eta_k|.
	\]
\end{enumerate}

\subsection*{Jadade koonduvus nendes}

\end{document}